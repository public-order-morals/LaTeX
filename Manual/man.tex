%(subject) (theme) (title)
\documentclass[11ptm]{jsarticle}

\usepackage{etoolbox}

\usepackage{array}

\usepackage{enumitem}

\usepackage{amsmath, newtxmath}

\usepackage[dvipdfmx]{graphicx}
%\usepackage[draft, dvipdfmx]{graphicx}

\usepackage[hang, small, bf]{caption}
\usepackage[subrefformat=parens]{subcaption}
\captionsetup{compatibility=false}

\usepackage{listings, jlisting}

\lstset{
  basicstyle={\ttfamily\small},
  frame=tbrl,
  breaklines=true,
%  language=C,
  lineskip=-0.5ex,
  tabsize=2
}

\usepackage{tabularx}

\usepackage{hyperref}
\usepackage{pxjahyper}

\makeatletter
\AtBeginDocument{
\let\c@figure\c@lstlisting
\let\thefigure\thelstlisting
\let\ftype@lstlisting\ftype@figure
}
\makeatother

\title{{\Huge POaM資産管理システム(仮称)\\操作マニュアル}\\第1版}
\author{24 高橋祥吾\\26 田桑大輔\\29 田中稀尋\\30 谷川僚}
\date{}

%%%%%%%%%%%%%%%%%%%
%%%%%%%%%%%%%%%%%%%
\begin{document}


\setcounter{page}{0}

\maketitle
\thispagestyle{empty}

\clearpage

\setcounter{page}{0}
\thispagestyle{empty}

\tableofcontents
\clearpage


%%%%%%%%%%
%\clearpage
\section{マニュアルとは}
\label{sec:マニュアルとは}
マニュアルには種類がある。
\begin{itemize}
  \item 操作マニュアル
  \item 業務マニュアル
  \item 障害対応マヌカハニー
  \item システムマニュアル
\end{itemize}\par
それぞれについて少し解説する。\par
操作マニュアルとは、システムの使い方、操作方法を記述するマニュアル。システムの起動や終了の仕方をはじめ、システムが備えるすべての機能の使い方、その機能を使う際の操作方法を詳細に記述。\par
業務マニュアルとは、システムを使った業務の進め方を記述するマニュアル。ひとつひとつの業務の流れ・手順を記述するとともに、流れ・手順のどの時点でシステムのどの機能を使うのかがわかるように記述する。機能の使い方や操作方法を記述する必要はない(操作マニュアルに任せる)。\par
障害対応マニュアルとは、システムに障害が生じたときの対応処理の方法を記述したもの。ただし、エンドユーザが自力で対応・解決できる範囲の障害だけに限る。要はQ$\&$A。\par
システムマニュアルとは、システムの仕組みや構造・全体的な構成などを基本からわかりやすく説明したもの。エンドユーザ向けであり、システム管理者やエンジニア向けのマニュアルのような詳細かつ専門的なものではなく、入門レベルの概要的なものになる。\\\par
我々が制作するのは授業的には操作マニュアルのみで十分だが、その他のマニュアルを制作してもソフトウェア工学の点数として評価される。また、障害対応マニュアルの一部を掲載すると丁寧。


%%%%%%%%%%
%\clearpage
\section{わかりやすい操作マニュアル制作のポイント}
\label{sec:わかりやすい操作マニュアル制作のポイント}
\url{https://www.science.co.jp/document_blog/26059/}

%%%%%
%\clearpage
\subsection{そもそも操作マニュアルとは}
\label{sec:そもそも操作マニュアルとは}
操作マニュアルとは文字通り、システムの操作方法を確認するためのドキュメント。マニュアルという立場上、操作マニュアルの多くは初心者や何も知らない顧客向けに作成される。よって、操作マニュアルの作成者は以下の点を意識する必要がある。
\begin{itemize}
  \item 初心者にも操作が理解しやすく記載されており、読後正しく操作ができる
  \item システムや機械を操作したあと、正しい操作だったかどうか、正解がわかる
  \item 困ったときに、すぐに該当のページに辿り着ける
\end{itemize}\par
世の中には、とりあえず操作手順を全部載せましたというマニュアルが存在している。分厚いだけのマニュアルは読み手の意欲を下げるので、ほとんど読まれることがない。また、直感的に操作できるシステムでは、作成しても操作マニュアルが読まれないケースがある。このような操作マニュアル未読のケースは、操作ミスや意図しない事故を発生させる可能性があり、大変危険である。\par
操作マニュアルの作成者は、先ほど挙げた3点を常に意識し、読了してもらえる操作マニュアルを完成しよう。

%%%%%
%\clearpage
\subsection{制作のポイント5つ}
\label{sec:制作のポイント5つ}
\begin{enumerate}
  \item 操作説明は網羅的に記載する
  \item 疑問点や注意事項は記載しておく
  \item 視覚的要素を用いる
  \item 操作の目的を記載する
  \item 操作の結果を記載する
\end{enumerate}
「Google検索からYahoo!のHPを開く」というケースで説明。

%%%%%
%\clearpage
\subsubsection{操作説明は網羅的に記載する}
\label{sec:操作説明は網羅的に記載する}
操作方法は細かいと良い。ただし冗長なものはわかりにくくなる。\par
操作内容の他にも、作業順、作業タイミング、操作者、条件などを記載するとよりわかりやすくなる。\par
\begin{lstlisting}
PC操作者 : Googleの検索バーに「yahoo」と打ち込み、Enterキー押下
検索結果が表示される
PC操作者 : 検索結果の一番上にある「Yahoo!JAPAN」をクリック
Yahoo!のHPを表示
\end{lstlisting}

%%%%%
%\clearpage
\subsubsection{疑問点や注意事項は記載しておく}
\label{sec:疑問点や注意事項は記載しておく}
操作にある程度慣れてくると疑問が湧くことがある。今回のケースで言えば、「Googleの検索履歴にYahoo!があるが、選択してもよいのか」など。このような疑問に対して、「入力履歴を選択しても問題ありません」などと記載しておくと、よりユーザビリティの高いマニュアルとなる。\par
また、注意事項や禁止操作も同時に記載すると良い。とはいえあらゆる疑問について記載していてはキリがないので、Q$\&$Aへ誘導するなどして、可読性とのバランスは取ったほうが良い。

%%%%%
%\clearpage
\subsubsection{視覚的要素を用いる}
\label{sec:視覚的要素を用いる}
文字だけでなく図や画像などを併せて操作マニュアルを記述しよう。

%%%%%
%\clearpage
\subsubsection{操作の目的を記載する}
\label{sec:操作の目的を記載する}
操作の目的を記載することで全体像の把握につながる。全体像の把握は作業効率の向上や作業重要性の理解につながる。\par
また、目次が目的ごとに並んでいれば、検索性が格段に上がる。自分のやりたい操作とページが瞬時に紐づくことで、作業が滞りなく進む。

%%%%%
%\clearpage
\subsubsection{操作の結果を記載する}
\label{sec:操作の結果を記載する}
今回のケースでいえば、以下のようなところ。
\begin{lstlisting}
  PC操作者 : Googleの検索バーに「yahoo」と打ち込み、Enterキー押下
  検索結果が表示される
\end{lstlisting}


%%%%%%%%%%
\clearpage
{\Large\bfseries\ 改変履歴}
\begin{table}[htbp]
  %\caption{}
  %\label{tb:}
  \centering
  \begin{tabularx}{\textwidth}{wc{0.2\linewidth}|wl{0.6\linewidth}|wl{0.2\linewidth}}
    日付       & 項目     & 担当者 \\
    \hline \hline
    2022/12/02 & 初版作成 & 高橋   \\
    %\hline
  \end{tabularx}
\end{table}


\end{document}
%%%%%%%%%%%%%%%%%%%
%%%%%%%%%%%%%%%%%%%

%
